\chapter{Related Work}\label{chap:relatedwork}
Give a brief overview of the work relevant for your thesis.

This thesis is building up from the work of Engesesser et al (2018) \cite{bolander2018better}. They investigated how a lazy agent, who had a preference against doing its own actions, compares to eager agents in planning problems. \extend{What did they find out? }
\extend{How is this different than my work? How does my work build up from their work? }

DEl Papiere,
Baltag und Moss
implicit coordination

distributed systems, verteilte systeme $->$ vor allem Tokens

Tokens have also made an appearance in distributed systems \todo{cite}. A distributed systems is a system where the components of the system are located apart from each other but they still have to communicate and coordinate their actions to archieve a common goal. This makes that field face some similar problems like the coordination of actions, the concurrency of the agents and the intransparency. The field also needs scalable solutions.

In this field \todo{specify the field again}, there are a few load balancing algorithms that try to equally distribute a task amongst several agents. Ray et al. (2012) \cite{ray2012execution} analysed the different existing load balancing algorithms like token routing, round robin, randomized, Central queuing and Connection mechanism. \\
Another use case in this field is the restictional use of a mutual resource that can only have a small number of users at a time. This can be archieved by using a token queue and a token semaphore, as from Makki et al. (1992) \cite{makki1992token}
