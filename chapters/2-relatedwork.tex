\chapter{Related Work}\label{chap:relatedwork}

%Give a brief overview of the work relevant for your thesis.
This thesis is building up from the work of Bolander et al (2018) \cite{bolander2018better}. They investigated how agent types impact the successfulness of implicitly coordinated plans. They introduced three agent types, lazy agents that have a preference against their own actions, eager agents, who have a preference for their own actions and optimally eager agents, that try to come up with optimal policies that reach the goal in the fewest number of steps. The problem with lazy agents is that they can produce deadlocks by expecting the other agent to act. This does not happen with eager agents, but eager agents can also potentially be over-eager and produce infinite executions by unintentionally working against each other. To solve this, they introduced optimally eager agents. In their work, they also had sequential action execution, but it could happen that two agents want to act at the same time, then the first agent who acts is not predefined. This could lead to the agents unintentionally sabotaging each other.

The formalism for epistemic planning in this thesis follows the work of Bolander et al. (2011) \cite{bolander2011epistemic}. They introduced the internal perspective of an agents view of the world to model partial observability in planning. The view of one agent can differ from other agents if the agent has different knowledge.

In recent work of Nebel et al. (2019) \cite{nebel2019implicitly} They investigated how implicitly coordinated plans without communications can succed. For this they used multi-agent path finding in which agents have to move to different destinations in a collision free manner. They first assumed that the destinations of the agents were common knowledge and the agents could not communicate and found that if agents are eager and have conservative replanning or plan optimally, the plans would succed. Then they dropped the assumption and researched how no communication worked with unknown destinations. They found that if the agents are eager and replan conservatively, they could also succed.

%DEL Papiere,
%Baltag und Moss
%\todo{implicit coordination}
%\todo{Bolander 2011}


%distributed systems, verteilte systeme $->$ vor allem Tokens

Tokens have also made an appearance in distributed systems for example in the work from Loucks et al. (1997) \cite{loucks1997system}. A distributed system is a system where the components of the system are located apart from each other but they still have to communicate and coordinate their actions to achieve a common goal. This makes that field face some similar problems like the coordination of actions, the concurrency of the agents and the in-transparency. The field also needs scalable solutions.

In the field of distributed systems, there are a few load balancing algorithms that try to equally distribute a task among several agents. Ray et al. (2012) \cite{ray2012execution} analyzed the different existing load balancing algorithms like token routing, round robin, randomized, Central queuing and Connection mechanism. \\
Another use case in this field is the restictional use of a mutual resource that can only have a small number of users at a time. This can be achieved by using a token queue and a token semaphore, as from Makki et al. (1992) \cite{makki1992token}

The way that they use the token differs a lot from this thesis' approach. Their tokens are used to restrict the access to a limited ressouce. Their agents do not plan with other agents and handing the token to the next player is usually done by a waiting queue. One focus of this thesis is that the agent who has the token gets to decide which agent will have the token next. This is not wanted in distributed systems because they want all agents to have the same rights of access to a ressource with no agent being strategically excluded.
