\chapter*{Abstract}

In this thesis, we are going to use epistemic planning for the decision making process in multi-agent planning with distributed knowledge. In multi-agent domains, the agents are independent with other agents actions causing changes not instigated by the agent. We extended dynamic epistemic logic to model tokens so the agent can decide the next acting agent. It is possible to solve planning tasks with joint goals without the agents having to negotiate about and commit to a centralized joint policy at plan time. We then evaluated this model by solving infinite executions and deadlocks.

\chapter*{Zusammenfassung}

In dieser Arbeit werden wir epistemische Planung für den Entscheidungsprozess in der Multi-Agent Planung mit verteiltem Wissen verwenden. In Multi-Agenten-Domänen sind die Agenten unabhängig von den Aktionen anderer Agenten, die Änderungen verursachen, die nicht vom Agenten eingeleitet wurden. Wir haben die dynamische epistemische Logik auf die Modellierung von Token erweitert, so dass ein Agent den nächsten aktiven Agenten entscheiden kann. Es ist möglich, Planungsaufgaben mit gemeinsamen Zielen zu lösen, ohne dass die Akteure zur Planzeit über eine zentralisierte gemeinsame Strategie verhandeln und sich dazu verpflichten müssen. Wir haben dann dieses Modell bewertet, indem wir unendliche Ausführungen und Deadlocks gelöst haben.
