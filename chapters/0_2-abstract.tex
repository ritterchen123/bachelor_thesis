\chapter*{Abstract}

In this thesis we are going to use epistemic planning for the decision making process in multi-agent planning with distributed knowledge. In multi-agent domains, each agent is independent from the actions of other agents resulting in changes not caused by the agent. We use dynamic epistemic logic to model tokens to avoid problems with deadlocks and infinite executions in implicit coordination.
Through tokens, an agent can decide the next active agent. It is possible to solve planning tasks with joint goals without the agents having to negotiate about and commit to a centralized joint policy at plan time. We then analyze our model by solving infinite executions and deadlocks.

\chapter*{Zusammenfassung}

In dieser Arbeit verwenden wir epistemische Planung für den Entscheidungsprozess in der Multi-Agenten Planung mit verteiltem Wissen. In Multi-Agenten-Domänen sind die Agenten unabhängig von den Aktionen anderer Agenten, die Änderungen verursachen, die nicht vom Agenten eingeleitet wurden.
Wir verwenden die dynamische epistemische Logik für die Modellierung von Tokens, um Probleme mit gegenseitigen Blockaden und unendlichen Ausführungen bei der impliziten Koordination zu vermeiden. Durch Tokens kann ein Agent den nächsten aktiven Agenten bestimmen. Es ist möglich, Planungsaufgaben mit gemeinsamen Zielen zu lösen, ohne dass die Agenten zur Planzeit über eine zentralisierte gemeinsame Strategie verhandeln und sich dazu verpflichten müssen. Schlussendlich evaluieren wir dann unser Modell, indem wir unendliche Ausführungen und Deadlocks lösen.
