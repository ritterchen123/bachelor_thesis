\chapter*{Abstract}

In this thesis, we are going to use epistemic planning for the decision making process in multi-agent planning with distributed knowledge. In multi-agent domains, the agents are independent with the actions of other agent causing changes not instigated by the agent. We have used dynamic epistemic logic to model tokens to avoid problems with deadlocks and infinite executions in implicit coordination.
Through tokens, an agent can decide the next active agent. It is possible to solve planning tasks with joint goals without the agents having to negotiate about and commit to a centralized joint policy at plan time. We then analyzed this model by solving infinite executions and deadlocks.

\chapter*{Zusammenfassung}

In dieser Arbeit werden wir epistemische Planung für den Entscheidungsprozess in der Multi-Agent Planung mit verteiltem Wissen verwenden. In Multi-Agenten-Domänen sind die Agenten unabhängig von den Aktionen anderer Agenten, die Änderungen verursachen, die nicht vom Agenten eingeleitet wurden.
Wir haben die dynamische epistemische Logik für die Modellierung von Tokens verwendet, um Probleme mit Deadlocks und infinite executions bei der impliziten Koordination zu vermeiden. Durch Tokens kann ein Agent den nächsten aktiven Agenten entscheiden. Es ist möglich, Planungsaufgaben mit gemeinsamen Zielen zu lösen, ohne dass die Akteure zur Planzeit über eine zentralisierte gemeinsame Strategie verhandeln und sich dazu verpflichten müssen. Wir haben dann dieses Modell evaluiert, indem wir unendliche Ausführungen und Deadlocks gelöst haben.
