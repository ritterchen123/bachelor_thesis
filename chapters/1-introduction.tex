\chapter{Introduction}\label{chap:introduction}

How would a perfect execution order look?

\todo{What is an execution order?}
\todo{Who are the agents?}

This thesis is trying to answer that question by looking at the different execution orders. In other pieces of writing \todo{which?} about epistemic planning the execution order is often unspecified, called an asynchronous execution order. This is easier to describe mathematically, but it also has some drawbacks. We are now going to have a look at the different execution orders, especially at the token based execution order.


%With the asynchronous execution order, every agent could intervene in the execution order at any time. This can potentially harm the execution and prevent the agents from reaching a goal. For example, if two agents have opposing goals, then one agent would always undo the actions of the other agent.
%To prevent this from happening, we want the agents to be able to decide their own execution order. Therefore, an agent should be able to specify the next agent. To model this, we used a token that the agents can pass on to the next agent. Only an agent with the token gets to act.


%\extend{How old is epistemic planning?}
The field of epistemic dynamic logic is relatively young. It started with the work from Plaza (1989) \cite{plaza1989logics} about ``Logics of public communications''.
Epistemic planning is taking regular planning task and enriching it with knowledge and belief. In regular planning, the world is simplified, so that everyone knows everything there is to know about that particular microverse. But this is only a very simplified model of the real world. In the real world, agents often have only little knowledge and can therefore make only limited calculations and beliefs on that world. To model that, we use epistemic logic. To show that the agents each have different knowledge that can change over time, we describe the logic as being dynamic.

%\extend{What is the future of epistemic planning?}

There are multiple ways this research could be used in the future, in the real world. One example is in autonomous driving. Two cars are standing across from each other on a bridge, only one car can pass. This problem could be solved with tokens.

%\extend{Why is this research relevant?}
%Autonomous driving, zusammenbrechen der kommunikation
%später dann auf Anwendung eingehen,
%Beispiel mit vier autos an einer Kreuzung, brauchen token

``Many applications, such as robotic warehouses, must coordinate a large number of agents simultaneously to carry out their specific tasks.'' This problem is usually described as the multi-agent path finding problem.

The thought of researching if a different order solves some problems is also done in multi agent path finding problems. Here, the problem gets more complex the more agents there are and it is very similar to the DEL approach.

This work is structured as follows:
First, we are going to highlight the related work to this thesis. Then, we are going to introduce the mathematical background, dynamic epistemic logic (DEL). This logic is explained using a specific example, introduced in the beginning. Finally we are going to present the research of this thesis.
