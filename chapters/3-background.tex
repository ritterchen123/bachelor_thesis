\chapter{Background}\label{chap:background}

Imagine the following game between two players, Anne and Bill. It  involves a lever placed in between them. The lever is upright in the middle position and has five positions in total. Anne thinks the lever should be pulled two positions to her side and Bill thinks the lever should be pulled two positions to his side.
Anne does not know Bill has a goal on his side, and Bill does not know Anne has a goal on her side.
This can be seen in figure \ref{leverexample}.\\
\input{figures/pictures/Hebelbild} \\

With the asynchronous execution order this game could be easily finished if Anne or Bill just pulled the lever two times in their direction. But just as easily this game could go on a really long time, if Anne and Bill pulled the lever alternating, once to the right and then to the left, then to the right again and so on. This could go on infinitely. With no set execution order, how can we prevent the game from going on infinitely?

\section{Dynamic Epistemic Logic}
In the following we will define and explain the core concepts of the Dynamic Epistemic Logic (DEL). DEL is a specific mathematical language used as the framework of this thesis.

The definitions are taken from the ``Better eager than lazy'' (2018) \cite{bolander2018better} paper, the ``A gentle introduction to DEL'' (2017) \cite{bolander2017gentle} and the book ``Dynamic epistemic logic'' from Ditmarsch \cite{Ditmarsch2007}.

Let $\mathcal{A}$ be a finite set of agents, from the example above this would be $\mathcal{A} =\{$Anne, Bill$\}$. Let $\mathcal{P}$ be a finite set of atomic propositions. Atomic propositions, like $p$ or $q$, describe some affairs that can be true or false.
The epistemic language $\mathcal{L}_{\text{KC}}$ is: \\
$$
\varphi ::= \top \ | \ \bot \ | \ p \ | \ \neg \varphi \ | \ \varphi \wedge \varphi \ | \ K_i \varphi \ | \ C\varphi
$$

with $p \in \mathcal{P}$ and $i \in \mathcal{A}$.
$\top$ describes $\varphi$ as being true, $\bot$ as false.
$K_i \varphi$ reads as ``Agent $i$ knows $\varphi$''. $C \varphi$ reads as ``it is common knowledge that $\varphi$ ''.
From the example before, the two agents, $a$ (Anne) and $b$ (Bill) have two goal positions, goal $p$ (lever to the left) and $q$ (lever to the right). Anne knows that one goal position is the left, but does not know the other: $K_a p \wedge \neg K_a q$ . Bill knows that one goal position is the right, but does not know the other position: $\neg K_b  p \wedge K_b q$.


Formulas are evaluated in epistemic models %\todo{not defined}
$$
\mathcal{M}=(W, (\sim_i)_{i \in \mathcal{A}}, V)
$$
with the domain $W$ being a nonempty finite set of worlds,
$\sim_i$ being an equivalence relation called the indistinguishably relation for agent $i \in \mathcal{A}$ and $V : P \rightarrow \mathcal{P}(W)$ assigning a valuation to each atomic proposition.

In the example above, Anne sees two worlds. One world where just the lever to the left is a goal, but also a world where another goal is the lever to the right. Those two worlds are indistinguishable for Anne. Let $w_1$ be the world where just the lever to the left is a goal and $w_2$ be the world where the goal is the lever to the left or to the right, then $w_1 \sim_a w_2$.
\[
\begin{tikzpicture}
  \draw (-2,0) node[world, label=below:{$w_1$}] (w1) {};
  \draw (0,0) node [desig] {}; % designation
  \draw (0,0) node[world, label=below:{$w_2$}] (w2) {};
  \draw (w1) -- node[above] {Anne} (w2);
\end{tikzpicture}
\]


For $W_d \subseteq W$, the pair $(\mathcal{M}, W_d)$ is called an epistemic state (or simply a state) and the worlds of $W_d$ are called designated worlds. A state is called global if $W_d=\{w\}$ for some world $w$ (called the actual world). We then often write $(\mathcal{M},w)$ instead of $(\mathcal{M},\{w\} )$. We use $S^{gl}(P,\mathcal{A})$ to denote the set of global states (or simply $S^{gl}$ if $P$ and $\mathcal{A}$ are clear from context). For any state $ s=(\mathcal{M}, W_d) $ we let $Globals(s)= \{ (\mathcal{M},w) | w \in W_d \} $.
A state $(\mathcal{M}, W_d)$ is called a local state for agent $i$ if $W_d$ is closed under $\sim _i$ (that is, if $w \in W_d$ and $w \sim _i v $, then $v \in W_d$).
Given a state $s=(\mathcal{M}, W_d)$ the associated local state of agent $i$, denoted $s^i$, is $(\mathcal{M},\{v|v\sim _i w \text{ and } w \in W_d\})$. Going from $s$ to $s^i$ amounts to a \textit{perspective shift} to the local perspective of agent $i$.

From the example above, Anne has the local state $(\mathcal{M},\{w_1,w_2\})$

Let $(\mathcal{M}, W_d)$ be a state on $P,\mathcal{A}$ with $\mathcal{M}=(W, (\sim_i)_{i \in \mathcal{A}}, V)$. For $i \in \mathcal{A}$, $p \in P$ and $\varphi, \psi \in \mathcal{L}_{\text{KC}}(P,\mathcal{A})$, we define truth as follows:
\begin{align*}
  &(\mathcal{M}, W_d) \models \varphi
    & &\text{ iff } \qquad
    (\mathcal{M},w)\models \varphi \text{ for all } w \in W_d \\
  &(\mathcal{M}, w) \models p
    & &\text{ iff } \qquad
    w \in V(p) \\
  &(\mathcal{M}, w) \models \neg \varphi
    & &\text{ iff } \qquad
    (\mathcal{M},w) \not\models \varphi \\
  &(\mathcal{M}, w) \models \varphi \wedge \psi
    & &\text{ iff } \qquad
    (\mathcal{M},w) \models \varphi \text{ and } (\mathcal{M},w) \models \psi \\
  &(\mathcal{M}, w) \models K_i \varphi
    & &\text{ iff } \qquad
    (\mathcal{M},v) \models \varphi \text{ for all } v \sim_i w \\
  &(\mathcal{M}, w) \models C \varphi
    & &\text{ iff } \qquad
    (\mathcal{M},v) \models \varphi \text{ for all } v \sim^* w \\
  &(\mathcal{M}, w) \models \top
    & &\text{        } \qquad
    \text{always} \\
  &(\mathcal{M}, w) \models \bot
    & &\text{        } \qquad
    \text{never}
\end{align*}
where $\sim^*$ is the transitive closure of $\bigcup_{i \in \mathcal{A}}\sim_i$.

Using Anne and Bill as an Example, Anne only knows $p$ is true. She cannot distinguish between $p,q$ and $p, \neg q$. Bill only knows $q$ is true, he cannot distinguish between $p,q$ and $\neg p,q$. A graphic representation of this would be the global state $s = (\mathcal{M}, w_2)$ with the nodes representing the worlds and the edges representing the indistinguishably relation. The circle around a node represent designated worlds.
\[
s=
\begin{tikzpicture}
  \draw (-2,0) node[world, label=below:{$w_1: p, \neg q$}] (w1) {};
  \draw (0,0) node [desig] {}; % designation
  \draw (0,0) node[world, label=below:{$w_2: p,q$}] (w2) {};
  \draw (2,0) node[world, label=below:{$w_3: \neg p, q$}] (w3) {};
  \draw (w1) -- node[above] {Anne} (w2);
  \draw (w2) -- node[above] {Bill} (w3);
\end{tikzpicture}
\]
We have the designated world $(M,w_2)$. THis designated world says nothing about the knowledge of Anne. She still cannot destinguish between if $q$ is true or not. $(M,w_2) \models \neg K_aq \vee \neg K_a \neg q$.
Performing a perspective sift on $s=(M,w_2)$ for Anne is $(M,w_2)^{\text{Anne}} =(M,\{w_1,w_2\})$. This perspective shift gives us $(M,\{w_1,w_2\})\not\models q$ and $(M,\{w_1,w_2\})\not\models \neg q$.

\subsection{Epistemic Actions and Product Updates}

An event model is $\mathcal{E} = \langle E, (\sim_i)_{i\in \mathcal{A}}, pre, \textit{eff}  \rangle$ where the domain $E$ is a non-empty finite set of events; $\sim_i \subseteq E \times E$ is an equivalence relation called the indistinguishably relation for agent $i$;
$pre:E \rightarrow \mathcal{L}_{KC}$ assigns a precondition to each event;
and $\textit{eff}:E \rightarrow \mathcal{L}_{KC}$ assigns a post condition, or effect to each event.
For all $e\in E$, $\textit{eff}(e)$ is a conjunction of literals, that means atomic propositions and their negations, including $\top$ and $\bot$.\\
For $E_d \subseteq E$, the pair $(\mathcal{E}, E_d)$ is called an epistemic action, or simply action and the events in $E_d$ are called a local action for agent $i$ when $E_d$ is closed under $\sim_i$. \\
Each event of an action represents a different possible outcome.
By using multiple events $e, e' \in E$ that are indistinguishable ($e \sim e' )$, it is possible to model only partially observable actions.\\
If the event model has $E=\{e\}$, we will write $\mathcal{E}=\langle pre(e), \textit{eff}(e)\rangle$.

The product update is used to specify the next state resulting from performing an action in a state.
Let a state $s = (\mathcal{M},W_d)$ and an action $a=(\mathcal{E},E_d)$ be given with $\mathcal{M}=\langle W,(\sim_i)_{i \in \mathcal{A}}, V\rangle $ and $\mathcal{E}=\langle E, (\sim_i)_{i \in \mathcal{A}},pre, \textit{eff} \rangle$
then the product update of $s$ with $a$ is defined as $s \otimes a = ((W',(\sim_i')_{i \in \mathcal{A}}, W_d'))$ where :
 \begin{itemize}
   \item a new world, emerging from the old world with all executable actions \\
   $W'=\{(w,e)\in W \times E ~|~ \mathcal{M}, w \models pre(e)\};$
   \item new worlds are indistinguishable if the old world and the events are indistinguishable \\
   $\sim_i'=\{((w,e),(w',e')) \in W'\times W' ~|~ w \sim_i w' \text{ and } e \sim_i e'\};$
   \item all propositions become true if the proportions occurred in the effect of the event or the proposition was positive in the world bevore and did not appear negative in the effect \\
   $V'(p) = \{ (w,e) \in W' ~|~ \textit{eff}(e) \models p \text{ or } (\mathcal{M},w \models p \text{ and } \textit{eff}(e)\not \models \neg p)\};$
   \item designated worlds emerge from designated predecessor worlds and designated events \\
   $W_d' = \{ (w,e) \in W' ~|~ w \in W_d \text{ and } e \in E_d\}$.
 \end{itemize}
$a=(\mathcal{E}, E_d)$ is applicable in $s=(\mathcal{M},W_d)$ if for all $w \in W_d$ there is an event $e \in E_d$ so that $(\mathcal{M},w) \models pre(e)$.

\todo{Announcment action}

\todo{Sensing Action}

\section{Planning tasks}


A planning task $\Pi = \langle s_0, A, \omega, \gamma \rangle$ consists of a global state $s_0$ called the \textit{initial state}; a finite set of actions A; an owner function $\omega: A \rightarrow \mathcal{A}$; and a \textit{goal formula} $\gamma \in \mathcal{L}_{KC}$.

Consider a verson of the lever problem from before. For simplicity, in this example there is only one player and the lever can only be pulled once. The planning task $\langle s_0, \{ a_1 \} , \omega, p \rangle$ consists of the initial state $s_0 = $
\begin{tikzpicture}
  \draw (0,0) node [desig] {}; % designation
  \draw (0,0) node[world, label=below:{$\neg p$}] (w1) {};
\end{tikzpicture}
with the lever being in the upright position. The action $a_1$ =
\begin{tikzpicture}
  \draw (0,0) node [desig] {}; % designation
  \draw (0,0) node[world, label=below:{$e_1: \langle \top, p \rangle $}] (w1) {};
\end{tikzpicture}
has the owner $\omega(a_1) = 1$ (player 1). Everything is fully observable for the agent. The intuitive  solution should prescribe the action $a_1$ to agent 1, pulling the lever to the right.

\[
\begin{tikzpicture}
  \draw (0,0) node [desig] {}; % designation
  \draw (0,0) node[world, label=below:{$w_1: \neg p$}] (w1) {};
\end{tikzpicture}
%
~\otimes~
%
\begin{tikzpicture}
  \draw (0,0) node [desig] {}; % designation
  \draw (0,0) node[world, label=below:{$e_1: \langle \top, p \rangle $}] (w1) {};
\end{tikzpicture}
%
~=~
%
\begin{tikzpicture}
  \draw (0,0) node [desig] {};
  \draw (0,0) node[world, label=below:{$(w_1, e_1): p$}] (w1e1) {};
\end{tikzpicture}
\]


A policy $\pi$ for $\Pi = \langle s_0, A, \omega, \gamma \rangle$ is a partial mapping $\pi: S^{gl} \hookrightarrow \mathcal{P}(A)$ such that:
\begin{enumerate}
  \item Applicability\\
    We require actions to be applicable in all states they are assigned to: \\
    for all $a \in S^{gl}, a \in \pi(s): a$ is applicable in $s$.
  \item Uniformity \\
    If the policy $\pi$ prescribes some action $a$ to agent $i$ in state $s$ and agent $i$ cannot distinguish $s$ from some other state $t$, then $\pi$ has to prescribe the same action $a$ for $i$ in $t$ as well: \\
    for all $s,t \in S^{gl} $ such that $ s^{\omega(a)} = t^{\omega(a)}, a \in \pi(s): a \in \pi(t)$ \\

    \todo{Counterexample with queen}

  \item Determinism \\
    We require $\pi$ to be unambiguous for all agents in the sense that in each state $s$ where an agent $i$ is supposed to act according to $\pi$, $\pi$ will always prescribe the same action for agent $i$.
\end{enumerate}

The properties uniformity and applicability together imply knowledge of preconditions, the property that in each state, an agent who is supposed to perform a particular action must also know that the action is applicable in that state.

We also must allow policies to sometimes prescribe multiple actions of different owners to the same state. This is because the set of indistinguishable states can differ between the agents. To characterize the different outcomes of agents acting according to a common policy, we define the notion of policy executions.

An execution of a policy $\pi$ from a global state $s_0$ is a maximal (finite or infinite) sequence of alternating global states and actions $(s_0, a_1, s_1, a_2, s_2,...)$, such that for all $ m \leq 0$
\begin{enumerate}
  \item $a_{m+1} \in \pi(s_m)$ and
  \item $s_{m+1} \in Globals(s_m \otimes a_{m+1})$
\end{enumerate}
An execution is called successful for a planning task $\Pi = \langle s_0, A, \omega, \gamma \rangle$, if it is a finite execution $(s_0, a_1, s_1,...,a_n, s_n)$ such that $s_n \models \gamma$.

\extend{Example: Anna lets Bill pull the lever to the right. Globally this makes sense, since that is Bills goal. But individually, this makes no sense at all because Anne doesent know Bills goal.}

\todo{Geht das hier etwas schöner?}
We now want to restrict our focus to policies that are guaranteed to achieve the goal after a finite number of steps. More formally, all of their executions must be successful. As in nondeterministic planning, such policies are called strong (Cimatti et al. 2003 \cite{cimattietal}) \todo{Quelle lesen}.

For a planning task $\Pi = \langle s_0, A, \omega, \gamma \rangle$, a policy $\pi$ is called strong if $s_0 \in \text{Dom}(\pi) \cup \{s \in S^{gl} ~|~ s \models \gamma\}$ and for each $s \in \text{Dom}(\pi)$, any execution of $\pi$ from $s$ is successful for $\Pi$. A planning task is called solvable if a strong policy for $\Pi$ exists.
For $ i \in \mathcal{A} $ , we call a policy $i$-strong if it is strong and  $Globals(s_0^i ) \subseteq \text{Dom}(\pi) \cup\{ s \in S^{gl} ~|~ s \models \gamma \}$.

When a policy is i-strong it means that the policy is strong and defined on all the global states that agent $i$ cannot indistinguish between. It follows directly from the definition that any execution of an $i$-strong policy from any of those initially indistinguishable states will be successful. So if agent $i$ comes up with an $i$-strong policy, agent $i$ knows the policy to be successful.

Sometimes the agents cannot coordinate their plans but rather have to come up with plans individually. These plans can differ a lot, the agents ofthen have different knowledge about the states, the actions and therefore the action outcomes. For this reason we will define a policy profile for a planning task $\Pi$ to be a family $(\pi_i)_{i \in \mathcal{A}}$ where each $\pi_i$ is a policy for $\Pi$. We assume actions to be instantaneous and executed asynchronously. This leads to the following generalization:

An execution of a policy profile $(\pi_i)_{i \in \mathcal{A}}$ is a maximal (finite or infinite) sequence of alternating global states and actions $(s_0, a_1, s_1,...)$, such that for all $m \leq 0$,
\begin{enumerate}
  \item $a_{m+1} \in \pi_i(s_m)$ where $i=\omega(a_{m+1})$ \\
    Note here the source of nondeterminism as a result from the possibility of multiple policies prescribing actions for their respective agents.
  \item $s_{m+1} \in Globals(s_m \otimes a_{m+1}) $ \\
    Here the source of nondeterminism is from the possibility of nondeterministic action outcomes.
\end{enumerate}


If all agents have one strong policy in common which all of them follow, then at execution time, the goal is guaranteed to be eventually reached. If, however, each agent acts on its individual strong policy, then the incompatibility of the individual policies may prevent the agents from reaching the goal, even tough each individual policy is strong.
